\documentclass{article}


%\usepackage{lineno}
%\linenumbers

\usepackage[margin=1in]{geometry}

\usepackage{mathtools}
\usepackage{amssymb}
\usepackage{bbding}

\usepackage{listings}

\usepackage{tikz}
\usetikzlibrary{automata, positioning, arrows.meta}


\newcommand{\entails}{\models}
\newcommand{\notentails}{\not\models}


\newcommand{\ma}{m$\mathcal{A}^*$}
\newcommand{\agents}{\ensuremath{\mathcal{AG}}}
\newcommand{\fluents}{\ensuremath{\mathcal{F}}}

\newcommand{\closed}{\ensuremath{c}}
\newcommand{\locked}{\ensuremath{l}}
\newcommand{\watching}{\ensuremath{w}}
\newcommand{\open}{\emph{open}}

\title{MECA Planner \input{../VERSION} \\
User Manual
}

\begin{document}

\maketitle

\section{The depl planning language}

Planning problems for the DEPL planner are specified in the doxastic/epistemic
planning language (depl). A single depl file describes both a planning domain
and a planning problem.

For the complete depl grammar, defined with the ANTLR meta-language, consult
\texttt{mecaPlanner/src/deplParser/Depl.g4}.

depl generally ignores whitespace and newlines, inline comments (anything
following ``\texttt{//}''), and block comments (``\texttt{/*\ldots*/}'').

A depl file contains the following sections, which must occur in the correct
order, and are required unless specified as optional:

\begin{itemize}
\item types
\item objects
\item agents
\item passive (optional)
\item constants (optional)
\item fluents
\item initially
\item goals
\item actions
\end{itemize}

\noindent
Each section begins with the section name, followed by
``\texttt{\{}'', 
then the section contents, and then
``\texttt{\}}''. 


\subsection{types section}

The \emph{types} section defines a type heirarchy relating all objects in the planning
problem.
Here is an example \emph{types} section:
\begin{verbatim}
types{
    Robot-Actor,
    Human-Actor,
    Actor-Locatable,
    Food-Locatable,
    Locatable-Object,
    Location-Object
}
\end{verbatim}

The content of the \emph{types} section consists of a comma-separated list of
type definitions. A type definition takes the form
``\texttt{Subtype - Supertype}'' (whitespace surrounding the ``\texttt{-}'' is
optional). The type ``\texttt{Object}'' is built-in, and
the type heirarchy must be constructed such that every type is a subtype of
``\texttt{Object}''.  A type name begins with an uppercase letter, followed
by any number of upper- or lower-case letters, integers, and underscores.

The depl parser will use the type heirarchy defined in this section for two main
purposes. First, it will provide type checking of objects as a guard against errors.
Second, as we will see in some later sections, it will automatically expand some
statements that contain types into a set of statements instead containing
objects of those types, providing a convenient short-hand. Note, however, that
the type-heirarchy is a parse-time entity only, no type information will be
available to the planner.


\section{objects}

The \emph{objects} section defines the objects in the planning problem.
Here is an example \emph{objects} section:
\begin{verbatim}
objects{
    robot1-Robot,
    human1-Human,
    pizza-Food,
    hotdog-Food,
    room1-Location,
    room2-Location,
}
\end{verbatim}

The content of the \emph{objects} section consists of a comma-separated list of
object definitions. An object definition takes the form ``\texttt{object -
Type}'' (whitespace surrounding the ``\texttt{-}'' is optional). Each type must
either be ``\texttt{Object}'', or be defined in the \emph{types} section.  An
object name begins with a lowercase letter, followed by any number of upper-
or lower-case letters, integers, and underscores.



\section{agents}

The MECA planner considers three types of \emph{agents}. The single
\emph{system} and arbitrarily-numberd \emph{environment} agents are specified in
this section.  The third type, \emph{passive} agents, are specified in the next
section.

An \emph{agent name} must be an object that has been defined in the
\emph{objects} section. An agent can have any type, but it is generally
convenient to create a type for objects that will be agents, for example 
as we do with the ``\texttt{actor}'' type in our running example.

The \emph{agents} section consists of a comma-separated list of agent
definitions. There must be at least one agent definition. There must be exactly
one \emph{system} agent, definitions, the rest must be \emph{environment} agent
definitions. A system agent definition contains only an agent name (i.e. an
object that has been defined in the \emph{objects} section). An environment
agent definition takes the form ``\texttt{name\{Model\}}'', where \texttt{name}
is an agent name and \texttt{Model} is the name of a model class. The model
class name must begin with an uppercase letter, followed by any number of
lower- and upper-case characters, integers, and underscores. The model name must
be the same as the name of a java class that extends
\texttt{mecaPlanner.models.Model} (the class should be defined in a
\texttt{.java} file within \texttt{mecaPlanner/src/mecaPlanner/models/}.

The order in which agents are defined determines the order in which they will
act.

Here is an example \emph{agents} section:
\begin{verbatim}
agents{
    robot1,
    human1{PizzaModel},
}
\end{verbatim}

This example defines a system agent, \texttt{robot1}, and a single environment
agent, \texttt{human1}. \texttt{robot1} will act first, followed by
\texttt{human1}, and then \texttt{robot1} again, etc. The planner will query
\texttt{PizzaModel} to determine the predicted actions of \texttt{human1}, and
will attempt to construct a plan that specifies the actions of \texttt{robot1}.




\section{passive (optional)}




\section{constants (optional)}
\section{fluents}
\section{initially}
\section{goals}
\section{actions}



\end{document}
