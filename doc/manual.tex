\documentclass{article}


%\usepackage{lineno}
%\linenumbers

\usepackage[margin=1in]{geometry}

\usepackage{mathtools}
\usepackage{amssymb}
\usepackage{bbding}

\usepackage{listings}

\usepackage{tikz}
\usetikzlibrary{automata, positioning, arrows.meta}


\newcommand{\entails}{\models}
\newcommand{\notentails}{\not\models}


\newcommand{\ma}{m$\mathcal{A}^*$}
\newcommand{\agents}{\ensuremath{\mathcal{AG}}}
\newcommand{\fluents}{\ensuremath{\mathcal{F}}}

\newcommand{\closed}{\ensuremath{c}}
\newcommand{\locked}{\ensuremath{l}}
\newcommand{\watching}{\ensuremath{w}}
\newcommand{\open}{\emph{open}}

\newcommand{\kdff}{\ensuremath{KD45_n}}
\newcommand{\sfive}{\ensuremath{S5_n}}
\newcommand{\kbone}{\ensuremath{KB1}}
\newcommand{\kbtwo}{\ensuremath{KB2}}

\lstset{basicstyle=\ttfamily,breaklines=true}

\title{MECA Planner \input{../VERSION} \\
\large User Manual
}

\begin{document}

\maketitle


Planning problems for the MECA Planner are specified in the doxastic/epistemic
planning language: \emph{depl}. A single depl file describes both a planning domain
and a planning problem.
The grammar of depl is defined with the ANTLR meta-language in the file:
\texttt{src/deplParser/Depl.g4}.


\section{Example}

We will use an example depl file, which is located at 
\texttt{problems/example.depl}.

This describes a planning domain involving two agents: a robot, whose behavior
will be determined by the planner, and a human, whose behavior is
estimated by a predictive model. There are two rooms and two hallways: each
hallway connects the two rooms. The first room contains a coin and both agents,
the second room contains a pizza. The robot does not know whether the coin lies
heads- or tails-up, althought the human does. The robot knows about the pizza in
the second room, but the human is unaware of the pizza.
The robot's goal is to know how the coin lies without the human knowing that the
robot knows this.
The robot can determine
how the coin lies by looking at it, but if the human is in the same room, she
will observe this and know that the robot knows. The human is hungry, and would
leave for the second room if she know about the pizza (according to the
predictive model).
A succesful plan is for the robot to announce to the human that there is pizza
in the other room, wait for the human to leave by either of the halls, and then
observe the coin. Here is the example depl file describing this problem:

\lstinputlisting{../problems/example.depl}



\section{Structure}

A depl file contains the following sections, which must occur in the correct
order, and are required unless specified as optional:

\begin{itemize}
\item types
\item objects
\item agents
\item passive (optional)
\item fluents
\item constants (optional)
\item initially
\item goals
\item actions
\end{itemize}

\noindent
Each section begins with the section name, followed by
``\verb|{|'', 
then the section contents, and then
``\verb|}|''. 


\subsection{Whitespace and newlines}
Whitespace and newlines are ignored,
as are, C-style inline comments (anything
following ``\texttt{//}'') and block comments (``\texttt{/*\ldots*/}'').


\subsection{types}

The \emph{types} section defines a type heirarchy relating all objects in the planning
problem.
The content of the \emph{types} section consists of a comma-separated list of
type definitions. A type definition takes the form
``\texttt{Subtype - Supertype}''.
The type ``\texttt{Object}'' is built-in, and
the type heirarchy must be constructed such that every type is a subtype of
``\texttt{Object}''.  A type name begins with an uppercase letter, followed
by any number of upper- or lower-case letters, integers, and underscores.

The depl parser will use the type heirarchy defined in this section for two 
purposes. First, it will provide type checking of objects as a guard against errors.
Second, it will automatically expand some
statements that contain types into a set of statements instead containing
objects of those types, providing a convenient short-hand. 
The type heirarchy is a strictly parse-time entity: no type information will be
available to the planner.


\subsection{objects}

The \emph{objects} section defines the objects in the planning problem.
The content of the \emph{objects} section consists of a comma-separated list of
object definitions. An object definition takes the form ``\texttt{object -
Type}''. Each type must
either be ``\texttt{Object}'', or be defined in the \emph{types} section.  An
object name begins with a lowercase letter, followed by any number of upper-
or lower-case letters, integers, and underscores.
Objects will be used as arguments to predicates to build a set of atomic
boolean fluents (propositions).


\subsection{agents}

The MECA planner uses three types of \emph{agents}. The single
\emph{system} agent and any number of \emph{environment} agents are specified in
this section.  The third type, \emph{passive} agents, are specified in the next
section. 
An \emph{agent} must be an object that has been defined in the
\emph{objects} section. An agent can have any type, but it is generally
convenient to create a type for objects that will be agents, for example 
as we do with the ``\texttt{actor}'' type in our running example.

The \emph{agents} section consists of a comma-separated list of agent
definitions. There must be at least one agent definition. There must be exactly
one \emph{system} agent, definitions, the rest must be \emph{environment} agent
definitions. A system agent definition consists of only an agent name.
An environment
agent definition takes the form ``\texttt{name\{Model\}}'', where \texttt{name}
is an agent name and \texttt{Model} is the name of a model class. The model
class name must begin with an uppercase letter, followed by any number of
lower- and upper-case characters, integers, and underscores. The model name must
be the same as the name of a java class that extends
\texttt{mecaPlanner.models.Model}, and the class should be defined in a
\texttt{.java} file \texttt{src/mecaPlanner/models/}.

This example defines a system agent, \texttt{robot1}, and a single environment
agent, \texttt{human1}.
The order in which agents are defined determines the order in which they will
act. thus \texttt{robot1} will act first, followed by
\texttt{human1}, and then \texttt{robot1} again, etc. The planner will query
\texttt{PizzaModel} to determine the predicted actions of \texttt{human1}, and
will attempt to construct a plan that specifies the actions of \texttt{robot1}.

The planner considers two main things with \emph{system} and \emph{environment}
agents. First, as discussed above, it considers their actions (as specified by
the planner for \emph{system} agents, as predicted by a model for
\emph{environment} agents. Second, it models their beliefs and knowledge, which
are represented as part of the states over which the planner searches.

The depl parser and the Meca planner will use the information defined in this
section for three
purposes. First, agents will be associated with actions. Second, and the model
assigned to each environment agent will be queried to predict that agent's
actions. Third, the epistemic and doxiastic state and action
systems will maintain representations of agents' mental states.



\subsection{passive}

This optional section defines \emph{passive} agents, whose beliefs and knowledge are
modeled by the planner (as with system and environment agents), but
who do not act (unlike system and environment agents).
The passive section consists of a comma-separated list of passive agent
definitions. A passive agent definition takes the same form as a system
agent definition: an agent name.



\subsection{fluents}

The \emph{fluents} section defines all fluent atoms that will be available to the planner. A
fluent atom consists of a predicate and a (possibly empty) orderd list of
arguments. Arguments are objects, as defined in the \emph{objects} section.

This section contains a comma-delimited list of fluent definitions. A fluent
definition taks the form \verb|name(p1,...,pn)|, where \verb|name| is the
predicate, which begins with a lowercase letter, followed by any number of
upper- and lower-case letters, integers, and underscores, and each element of
\verb|(p1,...,pn)| is an argument. An argument is
\emph{either} an object name \emph{or} a type name. If all arguments
of a fluent definition are object names, a single fluent is defined. If any
arguments are type names, the fluent definition is automatically expanded,
substituting all objects that are of the specified type(s), in all combinations, to
construct multiple fluents.



\subsection{constants}

This optional section defines constant (either true or false) atoms. This
\emph{constants} section consists of a comma-delimited list of constant
definitions, where a constant definition takes either the form
\verb|name(p1,...,pn)| (true), 
or
\verb|!name(p1,...,pn)| (false).
Automatic type-expansion is allowed as with fluent definitions.

If a constant is defined multiple times, its previously-defined values will be
overriden. Thus, we could use type-expansion to construct
a large number of false constants, and then
override some of them to be true. As an example, separate from our
running pizza-robot example (which does not use constants), consider a domain
having many rooms, some (but not most) of which are connected to each other.
A constants section specifing these constraints might look like this:

\begin{verbatim}
constants{
    !connected(Room, Room),
    connected(room1,room2),
    connected(room1,room3),
    connected(room3,room4),
    connected(room4,room5),
}
\end{verbatim}

Similarly to the type heirarchy, constant definitions are a parse-time entity.
The planner does not have access to them.
Wherever a defined constant is found within a depl file, it is replaced withe a
\emph{true} literal or a
\emph{false} literal.  Cosntants can be used to simplify and clarify some definitions,
especially action definitions. For example, with the given example
\emph{constants} section, we might define a \emph{move} action that transitions
between two rooms (called, perhaps, \verb|?from| and \verb|?to|, see the
\emph{actions} section below) only if they are connected, specifying as a
precondition that \verb"connected(?from,?to) | connected(?to,?from)".  If this
were done using fluents instead of constants,
the parser would generate (and give to the planner) an
action for every pair of rooms. As ther planner searched for a plan, it
would repeatedly consider each of these move actions, only to discover that the
preconditions are never satisfied for the vast majority of them. If
the \verb|connected| constraints are defined as constants, the parser determines that
the preconditions for most of the possible \emph{move} actions are constatnly
false, and only generates (and passes to the planner) actions for movement
between connected rooms.



\subsection{initially}

This section defines the start state (or possibly a set of start states) for the planner.
A state definition has two parts. First, worlds and their valuations are
defined. Then, per-agent knowledge and belief relations are defined over worlds.



Each world definition takes the form
\verb|w <- {f1,...,fn}|
where \verb|w| is a world name (used to reference the world when later
definingrelations) and each of \verb|f1| \ldots \verb|fn| is a fluent 
that should be true in that world (unlisted fluents are false).

B[human1] <- {(w1,w1)},

The parser will reject Kripke structures that to not satisfy \sfive (on the
knowledge relation) \kdff (on the belief relation), and
\kbone and \kbtwo (between the belief and knowledge relations).



\subsection{The \emph{goals} section}

The \emph{goals} section specifies the goals the planner tries to achieve.
This section contains a comma-delimited list of goal formulae. The parser and's
these together to construct a single goal formula. A goal $g$ takes the form:

$$
g \coloneq
\texttt{true}      \mid
\texttt{false}     \mid
f         \mid
(g)       \mid
\texttt{B}[a]g     \mid
g \& g     \mid
g | g     \mid
!g        \mid
!g        \mid
\texttt{timestep} e i
$$

\noindent
where $f$ is an (fluent or constant) atom, $a$ is an agent, 
$e \coloneq
\texttt{==} \mid
\texttt{!=} \mid
\texttt{<} \mid
\texttt{<=} \mid
\texttt{>} \mid
\texttt{>=}
$, and $i$ is an integer.

$P[i]g$ ($i$ believes possibly $g$) is provided as syntactic sugar for
$!B[i]!g$.

Other supported syntax includes $K[i]g$, $i$ knows $g$, and $Cg$, $g$ is common
knowledge. However, these or not really recommended. The knowledge relatin is
intended primarily for performing belief revision when false beliefs are
corrected, not to represent agent's beliefs. The implementation of common
knowledge is equivalent to ``everyone knows'', not infinite iteration of shared
belief.

An example goals section is:
\begin{verbatim}
goals{
    at(robot1,room1),
    timestep<=7
}
\end{verbatim}


\subsection{The \emph{actions} section}

\subsection{Optional syntax}

The depl syntax tries to be flexible.
Any
``\texttt{\&}''
(and) can instead be written as
``\texttt{\&\&}''.
Any
``\texttt{|}''
(or) can instead be written as
``\texttt{||}''.
Any
``\texttt{!}''
(not) can instead be written as
``$\sim$''.


\section{Test}
The best way to test a depl file is with the \verb|test| program.

\end{document}
