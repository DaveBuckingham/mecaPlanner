\documentclass{article}


%\usepackage{lineno}
%\linenumbers

\usepackage[margin=1in]{geometry}

\usepackage{mathtools}
\usepackage{amssymb}
\usepackage{bbding}

\usepackage{listings}

\usepackage{tikz}
\usetikzlibrary{automata, positioning, arrows.meta}


\newcommand{\entails}{\models}
\newcommand{\notentails}{\not\models}


\newcommand{\ma}{m$\mathcal{A}^*$}
\newcommand{\agents}{\ensuremath{\mathcal{AG}}}
\newcommand{\fluents}{\ensuremath{\mathcal{F}}}

\newcommand{\closed}{\ensuremath{c}}
\newcommand{\locked}{\ensuremath{l}}
\newcommand{\watching}{\ensuremath{w}}
\newcommand{\open}{\emph{open}}

\title{MECA Planner \input{../VERSION} \\
\large User Manual v0.1
}

\begin{document}

\maketitle

\section{The depl planning language}

Planning problems for the DEPL planner are specified in the doxastic/epistemic
planning language (depl). A single depl file describes both a planning domain
and a planning problem.

For the complete depl grammar, defined with the ANTLR meta-language, consult
\texttt{mecaPlanner/src/deplParser/Depl.g4}.


A depl file contains the following sections, which must occur in the correct
order, and are required unless specified as optional:

\begin{itemize}
\item types
\item objects
\item agents
\item passive (optional)
\item fluents
\item constants (optional)
\item initially
\item goals
\item actions
\end{itemize}

\noindent
Each section begins with the section name, followed by
``\verb|{|'', 
then the section contents, and then
``\verb|}|''. 


\subsection{The \emph{types} section}

The \emph{types} section defines a type heirarchy relating all objects in the planning
problem.
Here is an example \emph{types} section:
\begin{verbatim}
types{
    Robot-Actor,
    Human-Actor,
    Actor-Locatable,
    Food-Locatable,
    Locatable-Object,
    Location-Object
}
\end{verbatim}

The content of the \emph{types} section consists of a comma-separated list of
type definitions. A type definition takes the form
``\texttt{Subtype - Supertype}'' (whitespace surrounding the ``\texttt{-}'' is
optional). The type ``\texttt{Object}'' is built-in, and
the type heirarchy must be constructed such that every type is a subtype of
``\texttt{Object}''.  A type name begins with an uppercase letter, followed
by any number of upper- or lower-case letters, integers, and underscores.

The depl parser will use the type heirarchy defined in this section for two main
purposes. First, it will provide type checking of objects as a guard against errors.
Second, as we will see in some later sections, it will automatically expand some
statements that contain types into a set of statements instead containing
objects of those types, providing a convenient short-hand. Note, however, that
the type-heirarchy is a parse-time entity only, no type information will be
available to the planner.


\subsection{The \emph{objects} section}

The \emph{objects} section defines the objects in the planning problem.
Here is an example \emph{objects} section:
\begin{verbatim}
objects{
    robot1-Robot,
    human1-Human,
    pizza-Food,
    hotdog-Food,
    room1-Location,
    room2-Location,
}
\end{verbatim}

The content of the \emph{objects} section consists of a comma-separated list of
object definitions. An object definition takes the form ``\texttt{object -
Type}'' (whitespace surrounding the ``\texttt{-}'' is optional). Each type must
either be ``\texttt{Object}'', or be defined in the \emph{types} section.  An
object name begins with a lowercase letter, followed by any number of upper-
or lower-case letters, integers, and underscores.



\subsection{The \emph{agents} section}

The MECA planner considers three types of \emph{agents}. The single
\emph{system} and arbitrarily-numberd \emph{environment} agents are specified in
this section.  The third type, \emph{passive} agents, are specified in the next
section.

An \emph{agent name} must be an object that has been defined in the
\emph{objects} section. An agent can have any type, but it is generally
convenient to create a type for objects that will be agents, for example 
as we do with the ``\texttt{actor}'' type in our running example.

The \emph{agents} section consists of a comma-separated list of agent
definitions. There must be at least one agent definition. There must be exactly
one \emph{system} agent, definitions, the rest must be \emph{environment} agent
definitions. A system agent definition contains only an agent name (i.e. an
object that has been defined in the \emph{objects} section). An environment
agent definition takes the form ``\texttt{name\{Model\}}'', where \texttt{name}
is an agent name and \texttt{Model} is the name of a model class. The model
class name must begin with an uppercase letter, followed by any number of
lower- and upper-case characters, integers, and underscores. The model name must
be the same as the name of a java class that extends
\texttt{mecaPlanner.models.Model} (the class should be defined in a
\texttt{.java} file within \texttt{mecaPlanner/src/mecaPlanner/models/}.

The order in which agents are defined determines the order in which they will
act.

Here is an example \emph{agents} section:
\begin{verbatim}
agents{
    robot1,
    human1{PizzaModel},
}
\end{verbatim}

This example defines a system agent, \texttt{robot1}, and a single environment
agent, \texttt{human1}. \texttt{robot1} will act first, followed by
\texttt{human1}, and then \texttt{robot1} again, etc. The planner will query
\texttt{PizzaModel} to determine the predicted actions of \texttt{human1}, and
will attempt to construct a plan that specifies the actions of \texttt{robot1}.

The planner considers two main things with \emph{system} and \emph{environment}
agents. First, as discussed above, it considers their actions (as specified by
the planner for \emph{system} agents, as predicted by a model for
\emph{environment} agents. Second, it models their beliefs and knowledge, which
are represented as part of the states over which the planner searches.


\subsection{The \emph{passive} section}

This optional section defines \emph{passive} agents, whose beliefs and knowledge are
modeled by the planner (as with \emph{system} and \emph{environment} agents), but
who do not act (unlike \emph{system} and \emph{environment} agents).

The \emph{passive} section consists of a comma-separated list of \emph{passive} agent
definitions. A passive agent definition takes the same form as a \emph{system}
agent definition: an agent name, i.e. an object name as has been defined in the
\emph{objects} section.


\subsection{The \emph{fluents} section}

This section defines all fluent atoms that will be available to the planner. A
fluent atom has a name and a (possibly empty) orderd list of parameters. Parameters are
objects, as deinfed in the \emph{objects} section.

This section contains a comma-delimited list of fluent definitions. A fluent
definition taks the form \verb|name(p1 \ldots pn)|, where \verb|name| is the
fluent name, which begins with a lowercase letter, followed by any number of
upper- and lower-case letters, integers, and underscores, and each
$p\in$\verb|(p1 \ldots pn)| is a parameter definition. A parameter definition is
\emph{either} an object name \emph{or} a type name. If all parameter definitions
for a fluent definition are object names, a single fluent is defined. If any
parameter definitions are type names, the fluent definition is expanded,
substituting all objects that are of the specified type(s), in all combinations, to
construct multiple fluents. For example, given the previous example \emph{types}
and \emph{objects} example sections, the following two example
\emph{fluents} sections are equivalent:

\begin{verbatim}
fluents{
    at(robot1, room1),
    at(human1, room1),
    at(pizza,  room1),
    at(hotdog, room1),
    at(robot1, room2),
    at(human1, room2),
    at(pizza,  room2,
    at(hotdog, room2),
    human_hungry(),
    door_open(),
    robot_charged(),
}
\end{verbatim}

\begin{verbatim}
fluents{
    at(Locatable, Location),
    human_hungry(),
    door_open(),
    robot_charged(),
}
\end{verbatim}



\subsection{The \emph{constants} section}


This optional section defines constant (either true or false) atoms. This
\emph{constants} section consists of a comma-delimited list of constant
definitions, where a constant definition takes either the form
\verb|name(p1 \ldots pn)| (true), 
or
\verb|!name(p1 \ldots pn)| (false).
Automatic type-expansion is allowed as with fluent definitions.

If a constant is defined multiple times, its previously-defined values will be
overriden. Thus, we could use type-expansion to construct
a large number of false constants, and then
override some of them to be true. As an example, separate from our
running pizza-robot example (which does not use constants), consider a domain
having many rooms, some (but not most) of which are connected to each other.
A constants section specifing these constraints might look like this:

\begin{verbatim}
constants{
    !connected(Room, Room),
    connected(room1,room2),
    connected(room1,room3),
    connected(room3,room4),
    connected(room4,room5),
}
\end{verbatim}

Similarly to the type heirarchy, constant definitions are a parse-time entity.
The planner does not have access to them (the planner is concerned with change).
Anywhere a defined constant is found, it is replaced withe a true literal or a
false literal.  depl uses constants to simplify and clarify some definitions,
especially action definitions. For example, with the given example
\emph{constants} section, we might define a \emph{move} action that transitions
between two rooms (called, perhaps, \verb|?from| and \verb|?to|, see the
\emph{actions} section below) only if they are connected, specifying as a
precondition that \verb"connected(?from,?to) | connected(?to,?from)".  If this
were done using fluents instead of constants (and just being careful never to
change their values), the parser would generate (and give to the planner) an
aciton for \emph{every} pair of rooms. As ther planner searched for a plan, it
would repeatedly consider each of these move actions, only to discover that the
preconditions are never satisfied for the vast majority of them. If we defining
the \verb|connected| constraints as \emph{constants}, the parser determines that
the preconditions for most of the possible \emph{move} actions are constatnly
false, and only generates (and passes to the planner) actions for movement
between connected rooms.



\subsection{The \emph{initially} section}

This section d


\subsection{The \emph{goals} section}
\subsection{The \emph{actions} section}

\subsection{Notes}

The depl syntax tries to be flexible.
Any
``\texttt{\&}''
(and) can instead be written as
``\texttt{\&\&}''.
Any
``\texttt{|}''
(or) can instead be written as
``\texttt{||}''.
Any
``\texttt{!}''
(not) can instead be written as
``$\sim$''.
Whitespace and newlines are ignored.
as are, inline comments (anything
following ``\texttt{//}''), and block comments (``\texttt{/*\ldots*/}'').



\end{document}
